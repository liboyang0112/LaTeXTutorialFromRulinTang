\documentclass[english,a4paper,10pt]{article}
\usepackage{fontspec}
\usepackage[bookmarks=true,bookmarksopen=true,colorlinks=true,hyperfootnotes=false,filecolor=black,linkcolor=blue,urlcolor=blue,pdfauthor={Robert Blazek},pdftitle={pinouTikz},pdfsubject={package for strings of tokens},pdfkeywords={pinouTikz,latex,chip,pinout},pdfcreator={LaTeX}]{hyperref}
\usepackage[margin=1.9cm]{geometry}
\usepackage{libertine}
\usepackage[scale=0.85]{noto-mono}
\usepackage[bottom]{footmisc}
\usepackage{amsmath,amssymb,amsfonts,color,eurosym,xspace,babel,listings}
\usepackage{pinoutikz}
\makeatletter
\definecolor{@xs@bckgcolor}{rgb}{0.9,1,1}
\definecolor{@xs@keywordspinouTikz}{rgb}{0,0,0.9}
\definecolor{@xs@keywordslatex}{rgb}{1,0,0}
\definecolor{@xs@arguments}{rgb}{0,0,0}
\definecolor{@xs@comments}{rgb}{0.5,0.5,0.5}
\definecolor{violet}{rgb}{0.66,0,0.66}
\lstset{%
	language=[AlLaTeX]TeX,%
	morekeywords={numexpr,detokenize},%
	float=hbp,%
	basicstyle=\small\ttfamily,%
	identifierstyle=\color{@xs@arguments},%
	keywordstyle=\color{@xs@keywordslatex},%
	commentstyle=\itshape\color{@xs@comments},%
	columns=fixed,%
	tabsize=4,%
	frame=single,%
	extendedchars=true,%
	showspaces=false,%
	showstringspaces=false,%
	numbers=left,%
	numberstyle=\tiny\ttfamily,%
	breaklines=true,%
	breakindent=3em,%
	backgroundcolor=\color{@xs@bckgcolor},%
	breakautoindent=true,%
	captionpos=t,%
	xleftmargin=1em,%
	xrightmargin=1em,%
	lineskip=0pt,%
	numbersep=1em,%
	classoffset=1,%
	morekeywords={% macros and commands of pinouTikz
		PDIP,%
		PLCC,%
		SOP},%
	keywordstyle=\color{@xs@keywordspinouTikz},%
	classoffset=0}
\makeatother

\newcommand\guill[1]{"#1"}
\newcommand\argu[1]{$\langle$\textit{#1}$\rangle$}
\newcommand\ARGU[1]{\texttt{\{}\argu{#1}\texttt{\}}}
\newcommand\arguC[1]{\texttt{[}\argu{#1}\texttt{]}}
\newcommand\arguCC[2]{\texttt{[}\argu{#1}{,}\argu{#2}\texttt{]}}
\newcommand\arguD[1]{\texttt{(}\argu{#1}\texttt{)}}
\newcommand\texte[1]{\texttt{text}${}_{#1}$}
\newcommand\etoile{$\langle$\texttt{[*]}$\rangle$}
\newenvironment{Conditions}[1][1cm]%
{\begin{list}%
	{$\vartriangleright$}%
	{\setlength{\leftmargin}{#1}
	 \setlength{\itemsep}{0pt}
	 \setlength{\parsep}{0pt}
	 \setlength{\topsep}{2ptplus3ptminus2pt}
	}}%
{\end{list}}
\renewcommand\th{${}^\text{th}$\xspace}
\newcommand\US{syntax unit\xspace}
\newcommand\USs{syntax units\xspace}
\newcommand\pinouTikz{\textsf{pinouTikz}\xspace}
\newcommand\styleexemple{\small\baselineskip1.03\baselineskip\vskip\baselineskip\relax}
\newcommand\styleexercice{\footnotesize}
\newcommand\verbinline{\lstinline[basicstyle=\normalsize\ttfamily]}
\newcommand\colorise{\color{violet}}
\begin{document}
\setlength{\parindent}{0pt}
\begin{titlepage}
	\null\par\vfill
	\begin{center}
		\begin{minipage}{0.75\linewidth}
			\begin{center}
				\Huge\bfseries \pinoutikzname\par\vspace{5pt}
				\small v\pinoutikzversion\par\vspace{25pt}
				\normalsize User's manual
			\end{center}
		\end{minipage}
	\end{center}
	\vspace{1cm}
	\begin{center}
		Robert {\sc Blazek}\par\small
		\href{mailto:robert.h.blazek@gmail.com}{\nolinkurl{robert.h.blazek@gmail.com}}\par\vspace{5pt}
		%\xstringenglishdate
	\end{center}
	\vfill\hrulefill
	\begin{center}
		\begin{minipage}{0.85\linewidth}
			\noindent
			\hfill\textbf{\textit{Abstract}}\hfill{}\medskip\par
			This package which requires $\varepsilon$-\TeX{}, provides macros for creating pinout diagrams of chips. 

		\end{minipage}
	\end{center}
	\hrulefill\vfill{}
\end{titlepage}

\tableofcontents

\section{Introduction}
\subsection{Description}
This package defines macros for generating symbolic pinout diagrams for different package classes, such as DIP, PLCC, etc.

\subsection{Motivation}
Whoever has ever had to do with FPGA or MCUs (whether for living and leisure), it's just natural he or she might have been in a need to document some pins. So was my case and since I failed in finding any package in \LaTeX{} to suit my needs, I opted for creating one myself.

I hope others will find it as useful as it was to me and my colleagues.

This is my first latex package documentation ever - and since I hate reinventing the wheel - this manual has been based upon that of \textbf{xstrings} - with the courtesy of the author, of course.

\section{The macros}
For a better understanding, let's see first the macros with the simpler arguments possible. No special catcode, no exotic token, no control sequence either: only alphanumeric chars will be contained in the arguments.\medskip

In the following chapters, all the macros will be presented this way:
\begin{itemize}
	\item a short description of the operation;
	\item the operation under special conditions. For each conditions considered, the operation described has priority on that (those) below;
	\item finally, several examples are given. I tried to find them most easily comprehensible and most representative of the situations met in normal use.
\end{itemize}\smallskip

\textbf{Important}: in the following, a \argu{number} can be an integer written with numeric chars, a counter, or the result of an arithmetic operation made with the command \verbinline|\numexpr|.\smallskip

All the macros of \pinouTikz are displayed in {\makeatletter\color{@xs@keywordspinouTikz}red}.

\subsection{The pinout diagrams}
\subsubsection{\ttfamily\textbackslash PDIP}
\verbinline|\PDIP|\arguD{pincount}\ARGU{pinarray}
\smallskip

Draws a PDIP package with generic number of pins.

\begin{itemize}
	\item \argu{pincount} the  number of pins of a DIP package and should be an even number.
	\item \ARGU{pinarray} is a comma-separated list of pins  - each pin definition is as follows: \argu{pinnumber}/\ARGU{pinlabel}.
\end{itemize}

%% 4-pin example
%\begin{minipage}[t]{0.65\linewidth}
\begin{lstlisting}
\begin{figure}
  \centering
  \PDIP(4){%
  	1/{E},2/B,3/NC,4/C}
  \caption{NPN-Transistor, 4-pin PDIP package} \label{fig:X_DIP4}
\end{figure}
\end{lstlisting}%

%% 4-pin example
\begin{figure}[ht!]
  \centering
  \PDIP(4){%
  	1/{E},
  	2/B,
  	3/NC,
  	4/C
  	}%
  \caption{NPN-Transistor, 4-pin PDIP package} \label{fig:X_DIP4}
\end{figure}

%\end{minipage}\hfill
%% 8-pin example
%\begin{minipage}[t]{0.65\linewidth}
\begin{lstlisting}
\begin{figure}
  \centering
  \PDIP(8){%
  	1/CLK,
  	2/A,
  	3/B,
  	4/GND,
  	5/Y,
  	6/{\FormatPinLabel{~Y~}/RESET},
  	7/NC,
  	8/$V_{cc}$%
  	}
  \caption{TTL logic chip, 8-pin PDIP package} \label{fig:X_DIP8}
\end{figure}
\end{lstlisting}%

%% 8-pin example
\begin{figure}[ht!]
  \centering
  \PDIP(8){%
  	1/CLK,
  	2/A,
  	3/B,
  	4/GND,
  	5/Y,
  	6/{\FormatPinLabel{~Y~}/RESET},
  	7/NC,
  	8/$V_{cc}$%
  	}
  \caption{TTL logic chip, 8-pin PDIP package} \label{fig:X_DIP8}
\end{figure}
%\end{minipage}\hfill

%% 14-pin example
\begin{figure}[ht!]
  \centering
  \PDIP(14){%
  	1/A1,									
  	2/B1,									
  	3/\FormatPinLabel{\#1 AND/~OR~/GPIO1},		
  	4/Y1,									
  	5/C1,									
  	6/\FormatPinLabel{\#2 AND/~OR~/GPIO2},		
  	7/GND,								
  	8/PCLK,								
  	9/PDAT,								
  	10/A2,								
  	11/B2,								
  	12/\FormatPinLabel{\#2 INV/~SME~/GPIO3},	
  	13/C2,								
  	14/$V_{cc}$}							
  \caption{Generic programmable TTL logic chip, 14-pin PDIP package} \label{fig:X_DIP14}
\end{figure}

%% 14-pin example
\begin{minipage}[t]{0.65\linewidth}
\begin{lstlisting}
\begin{figure}[ht!]
  \centering
  \PDIP(14){%
  	1/A1,										
  	2/B1,										
  	3/\FormatPinLabel{\#1 AND/~OR~/GPIO1},			
  	4/Y1,										
  	5/C1,										
  	6/\FormatPinLabel{\#2 AND/~OR~/GPIO2},			
  	7/GND,									
  	8/PCLK,									
  	9/PDAT,									
  	10/A2,									
  	11/B2,									
  	12/\FormatPinLabel{\#2 INV/~SME~/GPIO3},		
  	13/C2,									
  	14/$V_{cc}$}								
  \caption{Generic programmable TTL logic chip, 14-pin PDIP package} \label{fig:X_DIP14}
\end{figure}
\end{lstlisting}%
\end{minipage}\hfill

\subsubsection{\ttfamily\textbackslash TQFP}
\verbinline|\TQFP|\arguD{pinnumber}\ARGU{pinarray}
\smallskip

Draws a TQFP package with generic number of pins.

\begin{itemize}
	\item \argu{pincount} the  number of pins of a DIP package and should be an even number.
	\item \ARGU{pinarray} is a comma-separated list of pins  - each pin definition is as follows: \argu{pinnumber}/\ARGU{pinlabel}.
\end{itemize}

%% 32-pin TQFP example
\begin{minipage}[t]{0.65\linewidth}
\begin{lstlisting}
\begin{figure}[ht!]
  \centering
  \TQFP(32){%
  	1/{PD.0/RTX1},
  	2/{PA.0/STX1},
  	3/PA.1,										
  	4/PA.2,										
  	5/PA.3,										
  	6/PA.4,										
  	7/PA.5,										
  	8/GND,									
  	9/PA.6,
  	10/PA.7,
  	11/{PB.0/RTX1},
  	12/{PB.1/STX1},
  	13/PB.2,										
  	14/PB.3,										
  	15/PB.4,										
  	16/PB.5,										
  	17/PB.6,										
  	18/PB.7,										
  	19/\FormatPinLabel{PC.0/~ALE~/PLPBC0},
  	20/PC.1,										
  	21/PC.2,										
  	22/PC.3,										
  	23/PC.4,										
  	24/PC.5,										
  	25/PC.6,									
  	26/PC.7,
  	27/XTAL1/PD.3,
  	28/XTAL2/PD.4,
  	29/RST,
  	30/PD.1,										
  	31/PD.2,										
  	32/$V_{cc}$}								
  \caption{A generic MCU chip, 32-pin TQFP package} \label{fig:X_TQFP32}
\end{figure}
\end{lstlisting}%
\end{minipage}\hfill

%% 32-pin TQFP example
\begin{figure}[ht!]
  \centering
  \TQFP(32){%
  	1/{PD.0/RTX1},
  	2/{PA.0/STX1},
  	3/PA.1,										
  	4/PA.2,										
  	5/PA.3,										
  	6/PA.4,										
  	7/PA.5,										
  	8/GND,									
  	9/PA.6,										
  	10/PA.7,										
  	11/{PB.0/RTX1},
  	12/{PB.1/STX1},
  	13/PB.2,
  	14/PB.3,
  	15/PB.4,
  	16/PB.5,
  	17/PB.6,
  	18/PB.7,
  	19/\FormatPinLabel{PC.0/~ALE~/PLPBC0},
  	20/PC.1,
  	21/PC.2,
  	22/PC.3,
  	23/PC.4,
  	24/PC.5,
  	25/PC.6,
  	26/PC.7,
  	27/XTAL1/PD.3,
  	28/XTAL2/PD.4,
  	29/RST,
  	30/PD.1,
  	31/PD.2,
  	32/$V_{cc}$}
  \caption{A generic MCU chip, 32-pin TQFP package} \label{fig:X_TQFP32}
\end{figure}

\subsubsection{\ttfamily\textbackslash PLCC}
\verbinline|\PLCC|\arguD{pinnumber}\ARGU{pinarray}
\smallskip

Draws a PLCC package with generic number of pins.

\begin{itemize}
	\item \argu{pincount} the  number of pins of a DIP package and should be an even number.
	\item \ARGU{pinarray} is a comma-separated list of pins  - each pin definition is as follows: \argu{pinnumber}/\ARGU{pinlabel}.
\end{itemize}

%% 28-pin PLCC example
\begin{minipage}[t]{0.65\linewidth}
\begin{lstlisting}
\begin{figure}[ht!]
  \centering
  \PLCC(28){%
  	1/{PD.0/RTX1},
  	2/{PA.0/STX1},
  	3/PA.1,										
  	4/PA.2,										
  	5/PA.3,										
  	6/PA.4,										
  	7/PA.5,										
  	8/GND,									
  	9/PA.6,
  	10/PA.7,
  	11/{PB.0/RTX1},
  	12/{PB.1/STX1},
  	13/PB.2,										
  	14/PB.3,										
  	15/PB.4,										
  	16/PB.5,										
  	17/PB.6,										
  	18/PB.7,										
  	19/\FormatPinLabel{PC.0/~ALE~/PLPBC0},
  	20/PC.1,										
  	21/PC.2,										
  	22/PC.3,										
  	23/{XTAL1/PD.3},
  	24/{XTAL2/PD.4},
  	25/RST,
  	26/PD.1,										
  	27/PD.2,										
  	28/$V_{cc}$}								
  \caption{A generic MCU chip, 28-pin PLCC package} \label{fig:X_PLCC28}
\end{figure}
\end{lstlisting}%
\end{minipage}\hfill

%% 28-pin PLCC example
\begin{figure}[ht!]
  \centering
  \PLCC(28){%
  	1/{PD.0/RTX1},
  	2/{PA.0/STX1},
  	3/PA.1,										
  	4/PA.2,										
  	5/PA.3,										
  	6/PA.4,										
  	7/PA.5,										
  	8/GND,									
  	9/PA.6,										
  	10/PA.7,										
  	11/{PB.0/RTX1},
  	12/{PB.1/STX1},
  	13/PB.2,
  	14/PB.3,
  	15/PB.4,
  	16/PB.5,
  	17/PB.6,
  	18/PB.7,
  	19/\FormatPinLabel{PC.0/~ALE~/PLPBC0},
  	20/PC.1,
  	21/PC.2,
  	22/PC.3,
  	23/{XTAL1/PD.3},
  	24/{XTAL2/PD.4},
  	25/RST,
  	26/PD.1,
  	27/PD.2,
  	28/$V_{cc}$}
  \caption{A generic MCU chip, 28-pin PLCC package} \label{fig:X_PLCC28}
\end{figure}

%% Examples for PDIP packages
\begin{minipage}[t]{0.35\linewidth}
	\styleexemple
\end{minipage}%


%\bigskip\bigskip
\begin{center}
$\star$\par
$\star$\quad$\star$
\end{center}
\par\nobreak\bigskip\bigskip
That's all, I hope you will find this package useful!\par\nobreak
Please, send me an \href{mailto:robert.h.blazek@gmail.com}{email} if you find a bug or if you have any idea of improvement\ldots\par\nobreak\medskip
Robert Blazek
\end{document}